\section{Template}

\subsection{Die Konzepte Informationssystem, Dateisystem, Datenbank, Datenbankmanagementsysten, Datenbanksystem definieren und differenzieren}

\paragraph{Informationssystem:}
Informationssystem lässt Informationen fliessen. Es besteht aus der Wissensbank, Methodenbank und Datenbank welche über das Softwaresystem vernknüpft ist.
Wikipedia ist beispielsweise ein Informationssystem. Daten sind Informationen welche technisch Abgespeichert sind (festgefrohrene Informationen).

\paragraph{Dateisystem:}

Ein Dateisystem besteht aus der Datenbasis und Hilfsprogrammen. Die Anwender greifen direkt auf die Datenbasis zu. Abbildung \ref{fig:Dateisystem mit Anwenderinteraktion} visualisiert das Dateisystem.

\begin{figure}[H]
	\centering
	\includegraphics[width=0.55\columnwidth]{9_Template/Bilder/Placeholder.png}
	\caption{Text zum Bild.}
	\label{fig:Dateisystem mit Anwenderinteraktion}
\end{figure}

Anders ausgedrückt:

\begin{itemize}
\item System zur Verwaltung von Datenbeständen
\item Besteht aus Datenbasis und Verwaltungsprogramm
\item Wird als Einheit gekapselt
\end{itemize}

\subsection{Das Konzept der NoSQL-Datenbanken definieren und erklären}



Abbildung \ref{fig:Neue Datenbanktechnologien für Big Data} eine erweiterung der 3V's von Big Data. Mit folgenden Massnahmen werden die V's angemessen behandelt:


\begin{figure} [H]
\centering
\begin{minipage}{0.5\textwidth}
  \centering
  \includegraphics[width=0.95\linewidth]{9_Template/Bilder/Placeholder.png}
  \captionof{figure}{Text}
  \label{fig:Neue Datenbanktechnologien für Big Data}
\end{minipage}%
\begin{minipage}{0.5\textwidth}
  \centering
  \includegraphics[width=0.95\linewidth]{9_Template/Bilder/Placeholder.png}
  \captionof{figure}{Text}
  \label{fig:NoSQL Datenbankmodelle}
\end{minipage}
\label{fig: NoSQL Datenbanken}
\end{figure}

\subsection{Vor- und Nachteile von SQL und NoSQL}

Vorteile von SQL:
\begin{itemize}
	\item Sie sind mehr als reine Datenspeicher:
		\subItem{Mächtige deklarative Sprachkonstrukte,}
		\subItem{Schemata und Metadaten,}
		\subItem{User, Rollen und Security,}
		\subItem{Referenzielle Integrität,}
		\subItem{Trigger,}
		\subItem{Indexierung,}
		\subItem{Query-Optimizer, …}
	\item Know-How breit vorhanden, Investitionsschutz
\end{itemize}


\textcolor{red}{Text in Farbe...}

Referenzieren der Tabelle \ref{tab:UMA vs NUMA}.


\begin{table}[h]
\centering
\begin{tabular}{p{4cm} |p{5cm} |p{5cm} }
	\toprule
	\textbf{Eigenschaft} & \textbf{UMA = Unified Memory Architecture} & \textbf{NUMA = Non-Unified Memory Architecture}\\
	\midrule
	Zugriffszeit auf Speicher & Einheitlich für alle & Nicht einheitlich \\
	\midrule
	Zugriff auf Speicherstelle & mehrere lokale Addressräume  & globaler linearer Addressraum\\
	\midrule
	Komplexität & einfacher und schneller Kommunikationsmechanismus & etwas komplexer und langsamer \\
	\midrule
	Skalierbarkeit & limitiert durch exponentiell ansteigende Anzahl Verbindungen & gut, ist nicht limitiert durch anzahl CPU's \\
	\bottomrule
\end{tabular}
\caption{Gegenüberstellung von UMA zu NUMA.}
\label{tab:UMA vs NUMA}
\end{table}

\begin{equation} \label{eqn: amdahl's law}
T = a + \frac{\left(1-a\right)}{n} 
\end{equation}

Gleichung \ref{eqn: amdahl's law} ist Amdahls Gesetz, mit T = Ausführungszeit, a = nicht parallelisierter Anteil, n = anzahl CPU's.

Folgend noch eine url \href{https://tex.stackexchange.com/questions/495020/how-do-i-change-the-spellcheck-language-of-texmaker-to-us-english}{Link zum Spelling Sprache ändern}